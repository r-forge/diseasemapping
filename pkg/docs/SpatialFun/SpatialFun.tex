\documentclass{article}
\usepackage{graphicx}
\textwidth 6.0 true in
\textheight 9 true in
\topmargin -1.0 true in
\headsep 0.375 true in
\oddsidemargin 0.0 true in
\evensidemargin 0.5 true in

\title{The Spatial Model Functions}
\def\ffmc{\mbox{ffmc}}
\def\wind{\mbox{wind}}
\def\ROS{\mbox{ROS}}
\def\ISI{\mbox{ISI}}
\def\BUI{\mbox{BUI}}
\begin{document}
\maketitle

\large
\section{Introduction}

\section{Method}
\subsection{The Poisson Model}
The followings are the theoretical functions:

Let $Y_k$ denotes the number of cancer cases in group k.
$P_k$ denotes the number of people (population size).


We have
     \[Y_k \sim Poisson (P_k\lambda_k),\]  we won't put time under our consideration at the beginning.  %or the function should be Y_k ~ Poisson(\lambda_i T)
     \[\log(\lambda_k) = \mu +  x_k\beta + U_k + V_k,\]
where      $\mu$ is the intercept, it could be \log(P_k);

           $x_k$ is the covariates i.e income variables;
           
           $\beta$ is the log relative risk;
           
           $U_k$ is the spatial random variable and $U_k$ \sim $N(0, \sigma_u^2)$
           
           $V_k$ is the non-spatial random variable and $V_k$ \sim $N(0, \sigma_v^2)$

Therefore, we will fit a $glm$ (generalized linear model) as:
    \[model = glm( cases \sim offset(logpop) + factor(age) factor(sex) )\]



\subsection{The R Functions}
Currently the diseasemapping package contains $6$ functions:  

\begin{enumerate}
\item 

\begin{verbatim}
''formatCases'' <- function(casedata)
\end{verbatim}



We get the data sets from different kinds of sources. Therefore, we need to clean and format the data set
for the future manipulation. The function dealing with two scenarios: when there is  \texttt{group} column in the data set
or when there is  \small \texttt{AGE\_SEX\_GROUP} column in data set. Essentially, we want to add in two
columns : sex and age.
\large

Group is usually a column looks like:
\begin{table}[ht]
\begin{center}
\begin{tabular}{|c|c|}
\hline
 $M0\_5$ \\
 $M5\_9$ \\
 $M10\_14$ \\
  \vdots  \\
 $M85PLUS$ \\
\hline
\end{tabular}
\end{center}
\end{table}


AGE\_SEX\_Group is usually a column of numbers composed with $3$ digits:
Male are the numbers begin with $1$, the rest two digits are corresponding to the age.
i.e:

\begin{table}[ht]
\begin{center}
\begin{tabular}{|c|c|}
\hline
$100$  &  $M0\_4$ \\
$101$  &  $M5\_9$ \\
$102$  &  $M10\_14$ \\
\vdots  &  \vdots  \\
$117$  &  $M85PLUS$ \\
\hline
\end{tabular}
\end{center}
\end{table}

Female is the same, but starting at $200$.


\item
\begin{verbatim}
''formatPopulation'' <- function(popdata)
\end{verbatim}

This function is similar to the formatCases function.
In some of the population data set, there will be a group of the people who are
age $85$ and plus, marked as: \small $M85PLUS$ or $F85PLUS$ \large for male and female respectively.
The first part of the formatPopulation function is to change them to \small $M85\_89$ or $F85\_89$,
\large so that they will have the same format as the other sex and age group, and that will
simplifies our manipulation later.
The second part of the function is to reshape the population data set using the function
\texttt{reshape}. So that it changes the population data set from the wide format
to the long format, and it will have the same format as the case data set. This step make
the merge of case data set and population data set easier in the future calculations.



\item 
\begin{verbatim}
''getRates'' <-
function(casedata, popdata, formula, family=poisson,
minimumAge=0, maximumAge=100, S=c("M", "F"))
\end{verbatim}

This is the function we constructed to fit a Generalized Linear Model (\texttt{glm} function) to
the case data set and the population data set. Then we get out estimated coefficients for the future
prediction.
The argument \texttt{casedata} and \texttt{popdata} are the places where we put case data set
and population data set correspondingly. \texttt{formula} is the place we fit a formula function
in the \texttt{glm} model. i.e : $~age*sex $. \texttt{family} is the place we
choose what distribution we want to fit in the \texttt{glm} model. By default we set it to be Poisson model, it
can be binomial as well.
\texttt{$minimumAge=0$} and \texttt{$maximumAge=100$} are the arguments to control the sample's age range. 
By default it is $age \in ($0, 100$)$ , which includes all the cases. You can change the range to ($0$, $10$) 
for chose all the kids' data set.  \texttt{S} is in charge of control what kind of sex group we want to choose. 
By default it choose both the male and female.


\item 
\begin{verbatim}
''getSMR'' <-
function(model, population, cases=NULL, regionCode="CSDUID",
  regionCodeCases="CSD2006")
\end{verbatim}

Function \texttt{getSMR} is used to do the prediction using the model we get from the
\texttt{getRates} function. Eventually, it returns a shape file (if the population data set
we insert the function is a shape file) contains everything of the original population
data set and several columns : expected, logExpected, cases, SMR. expected are the expected
case numbers we estimated using the model we get from \texttt{getRates} function. logExpected
are the log values of the expected values. cases are the number of cases we got from the
case data set (if we fit a poisson distribution it will be the aggregate number of cases by
different Decent Areas. SMR is the rate of observed number of cases over expected number of cases.

The argument \texttt{model} is the glm model estimated coefficients we got from the \texttt{getRates} function.
\texttt{population} is the population data set, it can be a shape file. \texttt{cases} is the number of
case data set. \texttt{regionCodeCases} and \texttt{regionCode} are used to indicate the region for case data and
population data sets respectively.

\item 
\begin{verbatim}
''area'' <- function(sp)
\end{verbatim}
Function \texttt{area} is used to calculate the area of each region, return a vector of areas. The argument \texttt{sp} is a single 
spatial polygon object.
\item 
\begin{verbatim}
''mergeBugsData'' <- function(x, bugsSummary, by.x = NULL, newcol="mean", ...) 
\end{verbatim}
Function \texttt{mergeBugsData} is used to merge the results from the \texttt{bugs} function to the other data set i.e population data set.

The argument \texttt{x} is the data set we want the result from the \texttt{bugs} to merge to. \texttt{bugsSummary} is the simulation result
get from the \texttt{bugs} function. \texttt{by.x} is the common column in both the {x} and {bugsSummary} data set. \texttt{newcol} is the summary statistic that to be merged back to the data frame.   
\end{enumerate}


\subsection{The Examples}
We will take the ontario long cancer data set as an example. Due to confidential needs, we simulated the case (cancer) data set.  
For example, we have a case data set as:


library
\begin{table}[ht]
\begin{center}
\begin{tabular}{cccccc}
Year& $CSD2006$ & $CD2006$ & $PR2006$ & AGE\_SEX\_GROUP & Cases \\
$1999$&$3501005$&$3501$&$35$&$210$&$3$ \\
$1999$&$3501005$&$3501$&$35$&$213$&$1$ \\
$1999$&$3501005$&$3501$&$35$&$214$&$6$ \\
$1999$&$3501005$&$3501$&$35$&$215$&$1$ \\
$1999$&$3501005$&$3501$&$35$&$216$&$3$ \\
\end{tabular}
\end{center}
\end{table}


After applying the \texttt{formatCases} function, it will be:
\begin{table}[ht]
\begin{center}
\begin{tabular}{cccccccc}
 Year& $CSD2006$ & $CD2006$ & $PR2006$ & AGE\_SEX\_GROUP &Cases &sex &age    \\
 $1999$& $3501005 $ & $3501 $  &  $35 $     &     $109$   &  $1$ &  M  &$45$ \\
 $1999$& $3501005 $ & $3501 $  &  $35 $     &     $110$   &  $1$ &  M  &$50$ \\
 $1999$& $3501005 $ & $3501 $  &  $35 $     &     $111$   &  $3$ &  M  &$55$ \\
 $1999$& $3501005 $ & $3501 $  &  $35 $     &     $112$   &  $1$ &  M  &$60$ \\
 $1999$& $3501005 $ & $3501 $  &  $35 $     &     $113$   &  $2$ &  M  &$65$ \\
 \end{tabular}
\end{center}
\end{table}

So that we can see, there are two new columns: sex and age.

Now let's look at the \texttt{formatPopulation} function:

For example, we have a population data set as:

\begin{table}[ht]
\begin{center}
\begin{tabular}{cccccc}

  CSDUID   &     CSDNAME  & CSDTYPE & PRUID  & PRNAME & CDUID \\
 $3501005$ & South Glengarry  &    TP  & $35$ &Ontario & $3501$ \\
 $3501011$ & South Stormont   &   TP   & $35$ &Ontario & $3501$  \\
 $3501012$ &      Cornwall    &  CY    &$35$  &Ontario & $3501$  \\
 $3501020$ &  South Dundas    &  TP    &$35$  &Ontario & $3501$  \\
 $3501030$ &  North Dundas    &  TP    &$35$  &Ontario & $3501$  \\
 $3501042$ &  North Stormont  &    TP  &  $35$& Ontario&  $3501$ \\

\end{tabular}
\end{center}
\end{table}


After applying the \texttt{formatPopulation} function, it will be:
\begin{table}[ht]
\begin{center}
\begin{tabular}{cccccccc}
   CSDUID &    cases&\ldots & CDUID & \ldots &expected    &logExpected  &     SMR    \\
 $3501005$&    $28$& \ldots & $3501$& \ldots &$8.433922$  &  $2.132262$ &$3.319926 $ \\
 $3501007$&    $ 0$& \ldots & $3501$& \ldots &$      NA$  &  $      NA$ &$      NA $ \\
 $3501011$&    $24$& \ldots & $3501$& \ldots &$7.143624$  &  $1.966220$ &$3.359639 $ \\
 $3501012$&    $52$& \ldots & $3501$& \ldots &$28.673675$ &  $ 3.355979$&$ 1.813510$ \\
 $3501020$&    $22$& \ldots & $3501$& \ldots &$6.654331$  &  $1.895268$ &$3.306117 $ \\
 $3501030$&    $19$& \ldots & $3501$& \ldots &$5.855935$  &  $1.767456$ &$3.244572 $ \\
 \end{tabular}
 \end{center}
 \end{table}

There are four new columns added in: cases, this is the number we got from the case data set;
expected, this is the predicted number of cases we calculated using the glm function;
logexpected are the \mbox{log} values of the expected numbers; SMR is the rate of \[\frac {oberved(cases)}{expected(expected)}.\]

if we ran the following code:
\begin{verbatim}
> model <- getRates( casedata, popdata, ~age*sex, S=c("M","F"))
\end{verbatim}

The result of model will be the predicted coefficients of different age and sex groups:
\begin{verbatim}
> model

Call:  glm(formula = formula, family = family, data = newdata)

Coefficients:
(Intercept)         age5        age10        age15        age20        age25
  -13.41885     -0.07403      0.52666      0.87985      1.62710      2.35525
      age30        age35        age40        age45        age50        age55
    3.49147      4.32618      5.12292      5.49001      5.96051      5.79087
      age60        age65        age70        age75        age80         sexM
    7.03365      7.11180      7.22133      7.24461      7.10451      0.67377
  age5:sexM   age10:sexM   age15:sexM   age20:sexM   age25:sexM   age30:sexM
         NA           NA           NA     -1.08827     -0.20732     -1.65308
 age35:sexM   age40:sexM   age45:sexM   age50:sexM   age55:sexM   age60:sexM
   -0.81485     -1.05938     -0.60803     -0.59874           NA     -0.73841
 age65:sexM   age70:sexM   age75:sexM   age80:sexM
   -0.38252     -0.25474     -0.20245           NA

Degrees of Freedom: 28 Total (i.e. Null);  0 Residual
Null Deviance:      10680
Residual Deviance: -5.529e-14   AIC: 229.2


\end{verbatim}



\section{Summary and Future Development}

\section{Bibliography}


\end{document}



