\documentclass{article}\usepackage[]{graphicx}\usepackage[]{color}
% maxwidth is the original width if it is less than linewidth
% otherwise use linewidth (to make sure the graphics do not exceed the margin)
\makeatletter
\def\maxwidth{ %
  \ifdim\Gin@nat@width>\linewidth
    \linewidth
  \else
    \Gin@nat@width
  \fi
}
\makeatother

\definecolor{fgcolor}{rgb}{0.345, 0.345, 0.345}
\newcommand{\hlnum}[1]{\textcolor[rgb]{0.686,0.059,0.569}{#1}}%
\newcommand{\hlstr}[1]{\textcolor[rgb]{0.192,0.494,0.8}{#1}}%
\newcommand{\hlcom}[1]{\textcolor[rgb]{0.678,0.584,0.686}{\textit{#1}}}%
\newcommand{\hlopt}[1]{\textcolor[rgb]{0,0,0}{#1}}%
\newcommand{\hlstd}[1]{\textcolor[rgb]{0.345,0.345,0.345}{#1}}%
\newcommand{\hlkwa}[1]{\textcolor[rgb]{0.161,0.373,0.58}{\textbf{#1}}}%
\newcommand{\hlkwb}[1]{\textcolor[rgb]{0.69,0.353,0.396}{#1}}%
\newcommand{\hlkwc}[1]{\textcolor[rgb]{0.333,0.667,0.333}{#1}}%
\newcommand{\hlkwd}[1]{\textcolor[rgb]{0.737,0.353,0.396}{\textbf{#1}}}%
\let\hlipl\hlkwb

\usepackage{framed}
\makeatletter
\newenvironment{kframe}{%
 \def\at@end@of@kframe{}%
 \ifinner\ifhmode%
  \def\at@end@of@kframe{\end{minipage}}%
  \begin{minipage}{\columnwidth}%
 \fi\fi%
 \def\FrameCommand##1{\hskip\@totalleftmargin \hskip-\fboxsep
 \colorbox{shadecolor}{##1}\hskip-\fboxsep
     % There is no \\@totalrightmargin, so:
     \hskip-\linewidth \hskip-\@totalleftmargin \hskip\columnwidth}%
 \MakeFramed {\advance\hsize-\width
   \@totalleftmargin\z@ \linewidth\hsize
   \@setminipage}}%
 {\par\unskip\endMakeFramed%
 \at@end@of@kframe}
\makeatother

\definecolor{shadecolor}{rgb}{.97, .97, .97}
\definecolor{messagecolor}{rgb}{0, 0, 0}
\definecolor{warningcolor}{rgb}{1, 0, 1}
\definecolor{errorcolor}{rgb}{1, 0, 0}
\newenvironment{knitrout}{}{} % an empty environment to be redefined in TeX

\usepackage{alltt}

\usepackage{float}
\usepackage{framed}
\usepackage{subcaption}
\usepackage{amsmath, amsthm,amssymb}
\usepackage{parskip}
\usepackage{graphicx}
\usepackage{listings}
\usepackage{enumitem}

\def\bigr{{\mathcal{R}}}
\newcommand{\class}[1]{`\code{#1}'}
\newcommand{\fct}[1]{\code{#1()}}
\def\reml{{\text{reml}}}
\def\N{{\text{N}}}
\def\MVN{{\text{MVN}}}
\def\cov{{\text{COV}}}
\def\T{{\footnotesize {^{_{\sf T}}}}}
\def\E{\mathbb{E}} % expectation
\newcommand{\mathleft}{\@fleqntrue\@mathmargin0pt}


\author{Ruoyong Xu}

\date{preliminary draft}
\IfFileExists{upquote.sty}{\usepackage{upquote}}{}
\begin{document}

\title{a title}

\maketitle
\begin{abstract}
abstract

\end{abstract}


\bibliographystyle{plain}


\section{Introduction} 
introduction

\section{The linear geostatistical model}
We start with the linear geostatistical model:
%
\begin{gather*}
Y_{i}|U(s_i) \overset{ind}\sim \N(\lambda(s_{i}),\tau^2 ),\\
\lambda(s_{i})=X(s_i)\beta+U(s_{i}),\\
(U(s_1), \dots,U(s_n))^\T \sim \MVN (0,\Sigma),\\
\Sigma_{ij}=\cov \left[U(s_i), U(s_j)\right]=\begin{cases} 
\sigma^2 R(\|s_i-s_j\|/\phi;\kappa),& i\neq j .\\
\sigma^2, & i=j.
\end{cases}
\end{gather*}
%
The basic ingredients are the following:
$Y_i: i=1,\dots,n$ is the observation obtained at location $s_i$. 
Given $U(s_i)$, $Y_i$'s are mutually independently distributed with Gaussian distribution with observation variance or nugget effect $\tau^2$.  
$X(s_i)$ is a $1 \times p$ vector of explanatory variables at location $s_i$, 
$\beta=(\beta_1,\dots,\beta_p)^\T$ is the corresponding vector of regression parameters. 
$U=(U(s_1), \dots,U(s_M))^\T $ is an (isotropic) zero-mean Gaussian random field with covariance matrix $\Sigma$ and correlation function $R(\cdot)$. 
$\sigma^2$ is the spatial variance or variability in residual variation.  

Let $Y=(Y_1,\dots,Y_n)^\T$ be the vector of observations, and $X$ be an $n \times p$ design matrix, we rewrite the model in matrix format:
%
\begin{equation} \label{eq:1}
Y \sim \MVN (X\beta, \sigma^2 R(\phi,\kappa)+\tau^2I), 
\end{equation}
% 
write $V=R(\phi,\kappa)+\nu^2I$, and $\nu^2={\tau^2}/{\sigma^2}$, then
\begin{equation*}
Y \sim \MVN (X\beta, \sigma^2 V).
\end{equation*}

\subsection{Mat\'ern correlation function}
There are three widely used classes of correlation functions in geostatistics: the Mat\'ern family, the powered exponential family and the spherical family. In general, we favour the Mat\'ern family because of its flexibility, this family has correlation function
\begin{equation*}
R(d;\phi,\kappa)=\frac{2^{\kappa-1}}{\Gamma(\kappa)} (\sqrt{8\kappa} \frac{d}{\phi})^\kappa  K_\kappa(\sqrt{8\kappa}  \frac{d}{\phi}),  \quad \text{where} \quad \phi    \geq 0,  \kappa     \geq 0,
\end{equation*} 
$d =\|x-x'\|$ is the Euclidean distance between two spatial points, 
$\sigma^2$ the variance of the random field $U$. $\Gamma(\cdot)$ is the standard gamma function,  $\phi$ is the range parameter or scale parameter with the dimension of distance, it controls the rate of decay of the correlation as $d$ increases. $K_\kappa(\cdot)$ is the modified Bessel function of the second kind with order $\kappa$, $\kappa$ is the shape parameter which determines the smoothness of $U(x)$, specifically, $U(x)$ is $m$ times mean-square differentiable if and only if $\kappa > m$. For $\kappa=0.5$, the Mat\'ern covariance function reduces to the exponential covariance function $\sigma^2*\exp(-d/\phi)$, when $\kappa \rightarrow \infty$, $\Sigma(d) \rightarrow \sigma^2*\exp{-(\|d\|/\phi)^2}$, which is called the Gaussian covariance. Estimation of $\kappa$ is difficult, our experience is to choose the values of $\kappa$ from a discrete set, for example $\{0.5,1.5,2.5,\dots \}$(ref), and then maximize the log-likelihood function \eqref{profile} or \eqref{remlpro} over $\phi$ and $\nu^2$. When the number of location points is large, the computation for the Mat\'ern function ...... more stuff ....


\subsection{Box-Cox transformation}
% What are the Box-Cox power transformations? The inference on the transformations parameter?
The fit of the linear geostatistical model can often be improved by applying a transformation to the response variable $Y$. Asymmetric (skewed) random fields that mildly deviate from the Gaussian pdf can be modeled by means of the Box-Cox transformation \citep{box1964analysis}.
%\cite{cressie2015statistics} introduced the term trans-Gaussian model to refer to the linear geostatistical model with a Box-Cox transformed  response variable. 
For positive-valued response variable $Y$, the Box-Cox transformation has the form:


\subsection{Geometric anisotroy}


\end{document}



